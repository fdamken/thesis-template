\chapter{Conclusion}
% TODO: CONCLUSION

In this study, we explored the effectiveness of machine learning techniques for intrusion detection in cybersecurity. Our experimental evaluation revealed that deep \acp{NN} and \acp{SVM} are the most effective models for detecting attacks with high accuracy rates, while \acp{DT} and \acp{RF} had lower accuracies. Our results also showed that the performance of the models varied depending on the type of attack being detected, and that feature selection techniques can further improve the accuracy of the models.

These findings have important implications for the development of more accurate and efficient intrusion detection systems in cybersecurity. By using deep neural networks and support vector machines, developers can create systems that are better equipped to handle the complex and high-dimensional data generated by cyber attacks. Furthermore, by using feature selection techniques, developers can improve the accuracy of these systems and reduce the number of false positives and false negatives.

It is important to note that our study has some limitations. First, we only evaluated a limited set of machine learning models and feature selection techniques. Future research could explore additional models and techniques to further improve the accuracy of \acp{IDS}. Additionally, our experiments were conducted on a specific dataset, and the performance of the models may vary on other datasets.

Despite these limitations, our study provides important insights into the effectiveness of machine learning techniques for intrusion detection in cybersecurity. Our results can help guide the development of more accurate and efficient intrusion detection systems, ultimately leading to better protection against cyber attacks.
