\chapter{Introduction}  \IMRADlabel{introduction}
% TODO: INTRODUCTION

In recent years, the increasing reliance on technology and the internet has led to a rise in the number and severity of cyber attacks. The consequences of these attacks can be devastating, ranging from data breaches and financial losses to reputational damage and legal liabilities. In response, organizations have invested significant resources in developing and implementing \acp{IDS} to protect their networks and systems from potential threats.

Machine learning has emerged as a promising approach to improving the effectiveness of intrusion detection systems. By leveraging the power of algorithms and data analytics, machine learning techniques can identify and classify patterns in network traffic and system logs that may indicate malicious activity. However, while there has been considerable research on the use of machine learning for intrusion detection, there is still a need for comprehensive evaluations of the effectiveness and efficiency of different algorithms and approaches.

The goal of this thesis is to explore the effectiveness of various machine learning techniques for intrusion detection in cybersecurity. Specifically, we evaluate the performance of \acp{DT}, \acp{RF}, \acp{SVM}, and \acp{NN} using a publicly available dataset. We compare the results of these algorithms in terms of accuracy, detection rate, and false positives, and identify the strengths and weaknesses of each approach. Through this research, we aim to contribute to the ongoing effort to enhance the effectiveness of intrusion detection systems and improve cybersecurity.
