\chapter{Preliminaries}
% TODO: PRELIMINARIES

The field of cybersecurity has become increasingly important in recent years due to the growing number of cyber attacks targeting individuals, organizations, and governments. \Acp{IDS} play a critical role in protecting computer networks and information systems by identifying and alerting users of any malicious or unauthorized activities.

Machine learning algorithms are particularly well-suited for intrusion detection because they can learn from large datasets and detect patterns that are difficult to identify manually. In recent years, various machine learning techniques have been applied to intrusion detection, including decision trees, support vector machines, and deep neural networks. These techniques have shown promising results in detecting various types of attacks, such as \ac{DoS}, distributed \ac{DDoS}, and buffer overflow attacks.

To understand the effectiveness of these machine learning techniques for intrusion detection, it is important to have a solid understanding of basic concepts and terminology related to machine learning, data analysis, and cybersecurity. This includes understanding different types of machine learning algorithms, such as supervised and unsupervised learning, and the metrics used to evaluate their performance, such as accuracy, precision, recall, and F1 score. Additionally, it is important to understand the characteristics of different types of attacks, their patterns, and the methods used to detect them.

In this thesis, we aim to explore the effectiveness of machine learning techniques for intrusion detection in cybersecurity. Specifically, we will compare the performance of several machine learning algorithms in detecting different types of attacks, and identify the most effective techniques for improving the accuracy and efficiency of \acp{IDS}.
