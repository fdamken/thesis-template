\chapter{Related Work}
% TODO: RELATED WORK

The use of machine learning for intrusion detection has been a topic of interest for many researchers in recent years. Several studies have explored the effectiveness of various machine learning algorithms for this purpose.

One of the most widely used machine learning techniques for intrusion detection is the \ac{DT} algorithm. In a study by \citeauthor{wang2017intrusion}, Decision Trees were found to have high accuracy and fast processing times, making them a popular choice for intrusion detection~\autocite{wang2017intrusion}.

Another popular algorithm for intrusion detection is the \ac{RF} algorithm, which uses an ensemble of decision trees to improve accuracy and reduce the risk of overfitting. In a study by \citeauthor{mirza2019random}, \ac{RF} was found to outperform other machine learning algorithms in terms of accuracy and detection rate~\autocite{mirza2019random}.

\acp{SVM} have also been widely used for intrusion detection. In a study by \citeauthor{raza2017intrusion}, \acp{SVM} were found to have high accuracy and low false positive rates, making it an effective approach for intrusion detection~\autocite{raza2017intrusion}.

\acp{NN}, specifically deep learning algorithms, have also been explored for intrusion detection. In a study by \citeauthor{khan2019intrusion}, a deep learning approach was found to outperform traditional machine learning algorithms in terms of accuracy and detection rate~\autocite{khan2019intrusion}.

While these studies have provided valuable insights into the effectiveness of various machine learning algorithms for intrusion detection, there is still a need for comprehensive evaluations and comparisons of different approaches. In this thesis, we aim to build upon and extend the existing research by evaluating the performance of multiple machine learning algorithms using a publicly available dataset.
