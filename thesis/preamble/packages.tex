\makeatletter
% WORKAROUND: biblatex is incompatible with etextools (loaded by autonum) and
% throws an error if etextools is loaded. However, autonum takes care of
% ensuring compatibility with other packages by restoring the functionality of
% the errorness commands. Defining 'blx@noerroretextools' ignores the error
% biblatex throws and turns it into a warning.
% The warning reads as follows. You can safely ignore it.
%     Incompatible package 'etextools' loaded,(biblatex) no error is thrown
%     because you defined(biblatex) '\blx@noerroretextools'.
\@namedef{blx@noerroretextools}{}
\makeatother

% core
\usepackage[
    main=USenglish,  % main language is US English
    ngerman,         % secondary is German (for the affidavit)
]{babel}

% math
\usepackage{mathtools}    % basic tools, includes amsmath and must be loaded first
\usepackage{amssymb}      % extended mathematical symbols
\usepackage{amsthm}       % theorems (we essentially only use the proof environment, everything else is handled by thmtools)
\usepackage{physics}      % d/dv, differentials, etc.
\usepackage{siunitx}      % units and number typesetting
\usepackage{bm}           % bold math, also for greek letters
\usepackage{bbm}          % blackbord bold for numbers
\usepackage{thmtools}     % extended version of amsthm
\usepackage{thm-restate}  % allows restating theorems

% load algorithm2e for pseudocode before cleveref, otherwise referencing lines
% in pseudocode does not work properly and it references the algorithm number
\usepackage[
    ruled,          % enclose algorithms in horizontal lines
    algochapter,    % include chapter number
    vlined,         % mark blocks with vertical lines
    noend,          % omit 'end' keyword (Pythonic)
    linesnumbered,  % number lines
    resetcount,     % reset line number count at the beginning of each algorithm
]{algorithm2e}

% references (load order amsmath -> hyperref -> cleveref -> autonum is crucial)
\usepackage{hyperref}              % basic references
\usepackage[
    capitalize,  % write Figure, Table, etc. capitalized
    nameinlink,  % also cross-reference the name, e.g., [Section 1.1](...)
    noabbrev,    % do not abbreviate cross-reference names
]{cleveref}                        % better references
\usepackage{autonum}               % auto-numbering equations
\usepackage{etoolbox}

% other
\usepackage[
    backend=biber,       % use biber backend
    doi=false,           % do not include DOI in references, they lead to overfull hboxes
    sorting=none,        % sort by appearance in text
    style=numeric-comp,  % compact citation mode, e.g., [1-4] instead of [1, 2, 3, 4]
    url=false,           % do not include URL in references (still included for @online references), they lead to overfull hboxes
]{biblatex}                          % bibliography
\usepackage{booktabs}                % beautiful tables
\usepackage[autostyle]{csquotes}     % correct styling of quotes
\usepackage[useregional]{datetime2}  % date/time formatting
\usepackage[bottom]{footmisc}        % force footnotes to be below floats
\usepackage[all]{foreign}            % foreign abbreviations, e.g., \eg, \ie, \etc, etc.
\usepackage{enumitem}                % styling of list environments
\usepackage[
    nopatch=footnote,  % footnote patching is incompatible with hyperref
]{microtype}                  % fine-tuning of fonts
\usepackage[
    abbreviations,           % create abbreviations glossary
    symbols,                 % create symbol glossary
    nomain,                  % do not use main glossary
    nogroupskip,             % do not add a blank row between groups
    record,                  % use bib2gls
    section,                 % place glossaries in sections, not chapters
    stylemods=longbooktabs,  % load booktabs style
    shortcuts=ac,            % enable \ac, \acp, etc. as shortcuts
    toc=false,               % do not add glossaries to table of contents
]{glossaries-extra}                  % glossaries
\usepackage{glossary-mcols}
\usepackage{makecell}                % allows line breaks in table cells
\usepackage{multirow}                % table cells spanning multiple rows
\usepackage{stmaryrd}                % \lightning
\usepackage{subcaption}              % subfigures
\usepackage{svg}                     % to include SVG files
\usepackage[
    para,  % use inline list instead of list with line breaks for brevity
]{threeparttable}                    % for tables with footnotes
\usepackage{xspace}                  % intelligent space (after macros)

% TikZ
\usepackage{tikz}       % tikz itself
\usepackage{tikzscale}  % scaling tikz pictures
\usetikzlibrary{
    arrows.meta,       % arrow styling
    positioning,       % relative positioning
    shapes.geometric,  % geometric shapes
}
